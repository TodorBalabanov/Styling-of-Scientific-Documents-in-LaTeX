LaTeX е безплатен софтуер с отворен код за оформяне на документи. LaTeX не е текстообработваща програма, а е език за тагиране на документи.

Първоначално е създадено от Leslie Lamport и е базиран на машината за набор TeX, създадена от Donald Knuth. Често го се ползва названието TeX, което също означава LaTeX.

LaTeX е много добър избор за създаване и поддържане на документи в областта на науката или за техническа документация. Едно от най-силните предимства на LaTeX е възможността за представяне на математически формули. За студенти и учени, LaTeX е способен да произведе документи с много високо качество, като гарантира стабилност на крайния резултат.

Друга много силна страна на системата е възможността за създаване на референции, като таблици на съдържанието, фигурите, таблиците, листингите и други, в съчетание с възможности за автоматично номериране, изграждане на библиография и индекс на използваните думи.

Осен полезността за хора занимаващи се с наука LaTeX дава и много голяма степен на гъвкавост. Съществуват множество шаблони за книги, писма, циркулярни писма, презентации, музикален нотопис и много други. Отвореният характер на системата е позволила на стотици потребители да създадат хиляди шаблони, стилове и полезни инструменти. Всеки един от вас може да се присъедини в разширяването на възможностите, предоставени от тази система.

Фундаментална концепция в LaTeX е разделението между съдържание и оформление. Авторът не е толкова ангажиран в оформлението и може значително по-сериозно да се концентрира върху създаването на съдържание. В други системи за текстова обработка, авторите са много по-ангажирани с паралелно оформление на своето съдържание. При LaTeX този процес се управлява през тагове (команди), които бързо се научават от създателите на LaTeX документи. Чрез промяна на класовете документи и пакетите, бързо и лесно може да се променя цялостното оформление на документа.

Тъй като LaTeX е софтуер с отворен код, то този инструмент е достъпен за почти всякакви операционни системи, най-основните сред които са Windows, MacOS и Linux. Основните файлови формати, ползвани в LaTeX, са текстови, така че могат да се четат и редактират под всякакви операционни системи. Това позволява LaTeX да създаде почти идентични резултатни документи в различните операционни системи. Системата TeX има множество различни дистрибуции, но в това изложение ще се ползва MiKTeX в комбинация с Texmaker, под операционна система Windows. 

За да се поддържа високата преносимост на LaTeX между различните операционни системи, не се използва графичен потребителски интерфейс. Авторите могат да създават своите документи с всеки текстов редактор, който предпочитат. Има създадени множество текстови редактори, специално предназначени за TeX документи. Такъв редактор е Texmaker, който се поддържа на Windows, MacOS и Linux. Това го прави изключително подходящ за целите на настоящото изложение. Веднъж свикнал потребителят с интерфейса на Texmaker, под някоя от операционните системи, то ползването му под другите операционни системи е почти идентично. 

Основните резултатни документи, които LaTeX генерира са в PDF формат. Това прави резултатните документи изключително преносими и използваеми под всякакви операционни системи и хардуер. PDF е формат, който изглежда идентично на различните платформи и го прави универсален за обмен на информация. Също така, резултатните документи могат да са и в други файлови формати, като HTML, DVI, PostScipt, ePub и други. Всичко това позволява информацията да се представя както за печатно издаване, така и за онлайн съдържание.

Голямо предимство в LaTeX е, че изходните текстове не се съдържат в комерсиален файлов формат или във файлови формати, които имат склонността да се превръщат в морално остарели и за които се налага използването на специално създадени конвертори. Не на последно място, намалява се рискът от заразяване с вируси, нещо което е възможно във файловите формати на други софтуери за текстообработка.


