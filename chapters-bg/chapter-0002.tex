След успешното създаване на първоначален документ, може да се пристъпи към по-сериозно оформление на текста и детайлите около него. Ще бъде разяснено логическото форматиране, как LaTeX чете входната информация, как да се модифицират шрифтовете, оформление на полета, параграфи, подравняване и цитиране.

\section{Логическо форматиране}

В самият текст на LaTeX документа не се изпълнява физическо форматиране. Авторът не се грижи дали текстът е нормален, удебелен, наклонен, подчертан или нещо друго. За целите на оформлението се използва логическо форматиране. Документът има различни фрагменти, като заглавие, име на автор, секция, подсекция и други. Физическото форматиране се извършва от LaTeX според вида на компонента и предварително заредения шаблон. 

В много редки случаи, когато става въпрос за единични изключения е допустимо да се използва и физическо форматиране, но то значително затруднява последващата поддръжка на единно оформление в документа.

Добре оформеният LaTeX документ използва физическо форматиране само в командите, които указват изпълнението на физическото форматиране.

\section{Обработка на входящата информация}



\section{Боравене с шрифтове}



\section{Оформяне на зони и параграфи}



\section{Подравняване и цитиране}



\section*{Обобщение}



